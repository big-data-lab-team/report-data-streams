\begin{abstract}
The advantage of embedded systems and wireless technologies led to the rise of
\acrshort{iot}, especially the \acrfull{wsn}. \acrshort{wsn} consist of various
sensor-equipped devices connected by a wireless network such as \acrfull{ble}.
In general, the connected devices in \acrshort{wsn} have very limited resources
such as memory, processing capabilities, and energy. Power consumption is a
critical problem targeting connected objects, it affects the lifetime of the
connected objects which are commonly expected to work for a long time. Data
summarization is the key for energy-constrained connected devices, where
transmitting fewer data can reduce energy usage during transmission. Data
compression which is one of the summarization technique can both compress the
data stream and keep the data information. Many compression methods have been
proposed in previous research. However, most of them either not applicable to
connected objects because of memory limitation or only handle one-dimensional
streams, while data acquired in connected objects are often multi-dimensional.
\acrfull{ltc} is among the lossy stream compression methods that provide the
highest compression rate for the lowest CPU and memory consumption.
In this paper, we investigate the extension of LTC to higher dimensions. First,
we provide a formulation of the algorithm in an arbitrary vectorial space of
dimension $n$.  Then, we implement the algorithm for the infinity and Euclidean
norms,  in spaces of dimension 2D+t and 3D+t. We evaluate our implementation on
3D acceleration streams of human activities.  Results show that the 3D
implementation of \acrshort{ltc} can save up to 20\% in energy consumption for
low-paced activities, with a memory usage of about 100~B. Finally, we compare
our extension methods with the 3-degree and 5-degree polynomial regression
compression method in different dimensions. As a result, by using our data set,
the extension of LTC gives a higher compression ratio than the polynomial
regression method.
\end{abstract}
