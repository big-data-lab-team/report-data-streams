\begin{abstract}
The rise of embedded systems and wireless technologies led to the emergence of the
\acrshort{iot}, especially \acrfull{wsn}. \acrshort{wsn} consist of various
sensor-equipped devices connected by a wireless network such as \acrfull{ble}.
In general, connected devices in \acrshort{wsn} have very limited resources
such as memory, processing capabilities, and energy. Power consumption is a
critical problem targeting connected objects, as it greatly affects their lifetime 
and usability. Data
summarization is key for energy-constrained connected devices, as
transmitting fewer data can reduce energy usage during transmission. Data
compression, in particular, can compress the
data stream \TG{data streams should be mentioned before, when you introduce WSNs. You could also 
introduce multi-dimensional streams. The sentences before could be contracted} while 
preserving information to a great extent.
 Many compression methods have been
proposed in previous research. However, most of them are either not applicable to
connected objects, due to resource limitation, or only handle one-dimensional
streams while data acquired in connected objects are often multi-dimensional.
\acrfull{ltc} is among the lossy stream compression methods that provide the
highest compression rate for the lowest CPU and memory consumption.
In this thesis, we investigate the extension of LTC to multi-dimensional streams. First,
we provide a formulation of the algorithm in an arbitrary vectorial space of
dimension $n$.  Then, we implement the algorithm for the infinity and Euclidean
norms, in spaces of dimension 2D+t and 3D+t. We evaluate our implementation on
3D acceleration streams of human activities, on Neblina, the connected device \TG{use a better word}
developed by our partner Motsai.  Results show that the 3D
implementation of \acrshort{ltc} can save up to 20\% in energy consumption for
low-paced activities, with a memory usage of about 100~B. Finally, we compare
our method with polynomial regression
compression methods in different dimensions. As a result, by using our data set,
the extension of LTC gives a higher compression ratio than the polynomial
regression method \TG{for a lower memory and CPU consumption}.
\end{abstract}
