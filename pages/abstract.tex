\begin{abstract}

The rise of embedded systems and wireless technologies led to the emergence of
the \acrfull{iot}. Connected objects in \acrshort{iot} communicate with each
other by transferring data streams over the network. For instance, in \acrfull{wsn},
sensor-equipped devices use sensors to capture properties, such as temperature
or accelerometer, and send 1D or nD data streams to a host system. Power
consumption is a critical problem for connected objects that have to work
for a long time without being recharged, as it greatly affects their lifetime
and usability.
Data summarization is key for energy-constrained connected devices, as
transmitting fewer data can reduce energy usage during transmission. Data
compression, in particular, can compress the data stream while preserving
information to a great extent.
% \TG{data streams should be mentioned before, when you introduce WSNs. You could also 
% introduce multi-dimensional streams. The sentences before could be contracted} 
Many compression methods have been proposed in previous research. However, most
of them are either not applicable to connected objects, due to resource
limitation, or only handle one-dimensional streams while data acquired in
connected objects are often multi-dimensional. \acrfull{ltc} is among the lossy
stream compression methods that provide the highest compression rate for the
lowest CPU and memory consumption.
In this thesis, we investigate the extension of LTC to multi-dimensional
streams. First, we provide a formulation of the algorithm in an arbitrary
vectorial space of dimension $n$.  Then, we implement the algorithm for the
infinity and Euclidean norms, in spaces of dimension 2D+t and 3D+t. We evaluate
our implementation on 3D acceleration streams of human activities, on Neblina, a
module integrating multiple sensors developed by our partner Motsai.  Results
show that the 3D implementation of \acrshort{ltc} can save up to 20\% in energy
consumption for slow-paced activities, with a memory usage of about 100~B.
Finally, we compare our method with polynomial regression compression methods in
different dimensions. Our results show that our extension of LTC gives a higher
compression ratio than the polynomial regression method, while using less memory
and CPU.
\end{abstract}
