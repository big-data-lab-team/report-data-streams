\section{Lossless Compression}
%\subsection{LZ77}
\subsection{LZSS}
LZSS~\cite{storer1982data} is a lossless textual compression algorithm. It is a
derivative and improvement of LZ77~\cite{ziv1977universal} which processes
compression by using previous data as dictionary. There is a sliding window and
a forward buffer applied in both algorithm. In LZ77 compression algorithm, it
outputs the original alphabet with empty pointer, e.g.(0, 0)A, if on sub-string
in forward buffer is found in dictionary\cite{ziv1977universal}, but it might
happen that this character $A$ and next alphabet compose a string appearing in
dictionary. LZSS solves these problems by defining a parameter
\emph{MIN\_LENGTH}(size of output pointers) and only output pointers ($n$, $m$),
where $n$ indicates the position of the first character of the repeating string,
$m$ is the length of the string~\cite{storer1982data}, rather than ($n$, $m$)$C$
in LZ77, where $C$ is the first first character at sub-string in forward buffer
stop finding in dictionary. The compression process as blow:

\begin{enumerate}
    \item Find the longest string $str$ which also appears in sliding
    window(dictionary) in forward buffer.
    \item Compute pointers $n$ and $m$
    \item Check if the length of $str$ is bigger or equals MIN\_LENGTH. If True,
    it outputs pointers and data stream goes forward length of $str$, else outputs
    the first character in forward buffer and stream goes forward 1
    byte(character).
\end{enumerate}

Assume sliding window and forward buffer is no-limited, MIN\_length = 2, a
example of using LZSS compression is given: A string $aaBccDaacEaccFacac$ is
encoded into $aaBccD$(1, 2)$cE$(8, 2)$cF$(8, 2)(8, 2) also simple to compress
and decompress, and it only decodes by its encoding result(no need to send
dictionary).

\todo{add figure show the structure of LZSS}

\subsection{A-LZSS}

\subsection{LZW}
\subsection{S-LZW}

\subsection{LEC}
LEC algorithm was first proposed by Marcelloni et
al.~\cite{marcelloni2008simple}, which is a approximated version of
exponential-Golomb code~\cite{teuhola1978compression}. In
paper~\cite{marcelloni2008simple}, authors measured temperature and humidity by
sensors and compressed data by using LEC. It uses very small dictionary whose
size is determined by the number of the bits after ADC
converter~\cite{marcelloni2008simple,marcelloni2009efficient}.  For instance, in
a sensor node, a measure $m_i$ is gained and converted into numeral value $r_i$
represented on $R$ bit by ADC. For each new data point $r_i$, LEC compute the
difference of adjacent points $d_i$ = $r_i$ - $r_{i-1}$, in order to compute
$d_0$, the $r_{-1}$ equals the central value among $2^R$. The residue $d_i$
encoded by entropy encoder, and represented as a bit sequence in 2 parts $s_i |
a_i$ , where $s_i$ is the value of the bits needed to represent the difference
$d_i$, and $a_i$ represent $d_i$ based on special rule:
\begin{enumerate}
    \item If $d_i = 0$, $a_i$ is not present.    
    \item If $d_i > 0$, $a_i$ is the binary expression of $n_i$ low-order bits
    of the $d_i$
    \item If $d_i < 0$, $a_i$ is the binary expression of $n_i$ low-order bits
    of the two's complement representation of ($d_i$ - 1)
\end{enumerate}

In the paper~\cite{marcelloni2008simple}, the authors generated the $s_i$ from
$n_i$ by Huffman coding. Let's say the $s_i$ is the representation at $n_i$ in
the Huffman code, $n_i$ = $(\log{_2}{\left|{b_i}\right|} +
1)$~\cite{li2016temporal} and $n_i \leq R$. Table~\ref{table:LEC} is used in
paper~\cite{marcelloni2009efficient}. From the Table~\ref{table:LEC}, it
supports R+1 groups($n_i$) and each group $s_i$ contains $2^{n_i}$ numeral
value.  
\begin{table}[]
\begin{tabular}{|l|l|l|}
\hline
$n_i$ & $s_i$        & $d_i$                                  \\ \hline
0     & 00           & 0                                      \\ 
1     & 010          & -1, +1                                 \\
2     & 011          & -3, -2, +2, +3                         \\
3     & 100          & -7, ..., -4, +4, ..., +7               \\
4     & 101          & -15, ..., -8, +8, ..., +15             \\
5     & 110          & -31, ..., -16, +16, ..., +31           \\
6     & 1110         & -63, ..., -32, +32, ..., +63           \\
7     & 11110        & -127, ..., -64, +64, ..., +127         \\
8     & 111110       & -255, ..., -128, +128, ..., +255       \\
9     & 1111110      & -511, ..., -256, +256, ..., +511       \\
10    & 11111110     & -1023, ..., -512, +512, ..., +1023     \\
11    & 111111110    & -2047, ..., -1024, +1024, ..., +2047   \\
12    & 1111111110   & -4095, ..., -2048, +2048, ..., +4095   \\
13    & 11111111110  & -8191, ..., -4096, +4096, ..., +8191   \\
14    & 111111111110 & -16383, ..., -8192, +8192, ..., +16383 \\
\hline
\end{tabular}
\caption{The dictionary table used in the experiments~\cite{marcelloni2009efficient}}
\label{table:LEC}
\end{table}
 
\subsection{S-LEC}
S-LEC Algorithm is a extension and improvement of LEC Algorithm by Li et
al.~\cite{li2016temporal}. Same with LEC, it encodes residues in compression
process, because of the correlation characteristic of sensor stream, e.g. the
different of stream data unlikely be too large. In LEC, the result after
encoding has two parts $s_i$ and $ a_i$, and the $s_i$ part shell cause
information redundancy, if a set of residues are in same group which means
repeated group index $s_i$. S-LEC reduces the size of representation by
exploiting the correlation amount adjacent residues. The main idea is using a
extra two bits of sequential code $b_i$ to present the positional relationship
of groups of adjacent residues~\cite{li2016temporal}. The encoding result
$s_i|a_i$ is replaced by $b_i|s_i|a_i$, but the group index $s_i$ shell be
omitted if two adjacent residues are in same group or neighboring group. Assume
the number of the group in table is K, the $b_i$ is defined
as~\cite{li2016temporal}: 

\begin{enumerate}
    \item $b_i$ = 00, if $s_i$ = $s_{i-1}$
    \item $b_i$ = 01, if $n_i$ = $n_{i-1}$ - 1($n_{i-1} \geqslant 1$), or $n_i$
    = $n_{i-1}$ + 2($n_{i-1}$ = 0)
    \item $b_i$ = 10, if $n_i$ = $n_{i-1}$ - 1($n_{i-1} \leqslant K$), or $n_i$
    = $n_{i-1}$ - 2($n_{i-1}$ = K)
    \item $b_i$ = 11, otherwise. The representation $b_i|s_i|a_i$ is required. 
  \end{enumerate}

When the $b_i$ equals 11, the representation need more bits than LEC result,
because of extra $b_i$. S-LEC proposed some context situations to solve this
problem. In paper~\cite{li2016temporal}, the groups is divided into three
clusters: $C_1$ = {$n_i$|$i$ = 0, 1, 2, 3}, $C_2$ = {$n_i$|$i$ = 4, 5} and 
$C_3$ = {$n_i$|$i$ = 6, ..., K}. The idea of reducing group code as
follow~\cite{li2016temporal}:

\begin{enumerate}
    \item $s_i$ remove the first "1", If $n_{i-1} \in C_1$
    \item $s_i$ remove the first two "1"s, If $n_{i-1} \in C_2$
    \item $s_i$ remove the first three "1"s, If $n_{i-1} \in C_3$
\end{enumerate}
S-LEC gives better Compression ratio then LEC when they are in same SME(Square
Mean Error)~\cite{li2016temporal}. We need to save the table into memory when we
uses these two algorithms, and in order to improve compression ratio as much as
possible, it is necessary to encode $n_i$ by prefix code method base on the
elements distribution.

\subsection{TMT}
