\section{Conclusion}

% \TG{A transition with the previous chapter is really missing, here or in the conclusion
% of the previous chapter. You also need to explain why you are doing this:
% why do we need a multi-dimensional implementation, why you chose LTC, etc}

In this chapter, general lossless and lossy compression methods were presented.
The \acrshort{ltc} compression method fully meets our requirements because of
its low computational and space complexity. We wonder what is the result of
using \acrshort{ltc} algorithm to compress the streams from Neblina. However, a
problem is that \acrshort{ltc} compression method which appropriates for
connected objects is just been described for 1D streams, while streams acquired
by connected objects, such as acceleration or gyroscopic data, are often
multi-dimensional~\cite{li2018multi}. We extend \acrshort{ltc} to dimension $n$
and give implementation and a norm-independent expression of it in
Chapter~\ref{chap:ltc-extension}. In Chapter~\ref{chap:expsAndResults} we test
LTC n-dimension method on 3D acceleration streams. 
