\section{Introduction}


\todo{add follow content into Chapter 2}
Compression is a key technique to reduce the rate of radio 
transmission.  While in several applications lossless compression 
methods are more desirable than lossy compression techniques, in the 
context of IoT and sensor data streams, the measured sensor data 
intrinsically involves noise and measurement errors, which can 
be treated as a configurable tolerance for a lossy compression algorithm. 

Resource-intensive lossy compression algorithms such as the ones based on 
polynomial interpolation, discrete cosine and Fourier transforms, or 
auto-regression methods~\cite{lu2010optimized} are not well-suited for 
connected objects, due to the limited memory available on 
these systems (typically a few KB), and the energy consumption 
associated with CPU usage. Instead, compression algorithms need 
to find a trade-off between reducing network communications and 
increasing memory and CPU usage. As 
discussed in~\cite{zordan2014performance}, linear compression methods 
provide a very good compromise between these two factors, leading to 
substantial energy reduction.
