\section{Conclusion}
In this chapter, we give a notation for original LTC algorithm and a pseudo-code
with our notation and definitions. To generalize LTC to dimension $n$, the LTC
property has been re-written in Section~\ref{sec:ltc-property}. According
Section~\ref{sec:effect-norm}, we known that the n-dimensional LTC might get
different results in different norm. In Euclidean norm, it is more difficult to
check the intersection and record this inter section in memory. We introduce the
method of combination of plane sweep and bisection for intersection test in
Euclidean norm, but it spends more processing time and costs more memory to
record a sequence of n-ball than for the Infinity norm. In next chapter, we will
show the result of using n-dimensional LTC on the accelerometer data of human
activities. 
