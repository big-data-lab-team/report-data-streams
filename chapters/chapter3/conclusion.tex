\section{Conclusion}
In this chapter, we provided a formulation for the original \acrshort{ltc}
algorithm and a pseudo-code with our notation and definitions. To generalize
\acrshort{ltc} to dimension $n$, the \acrshort{ltc} property has been re-written
in Section~\ref{sec:ltc-property}. According to Section~\ref{sec:effect-norm},
we know that the n-dimensional \acrshort{ltc} might get different results when
different norms are used. With the Euclidean norm, it is more difficult to check
the intersection and record this intersection in memory. We introduce the method
of combination of plane sweep and bisection for intersection test in Euclidean
norm, but it spends more processing time and costs more memory to record a
sequence of n-balls than the method for the Infinity norm. In next chapter, we
will evaluate the performance of n-dimensional \acrshort{ltc} on accelerometer
data for human activities. 
