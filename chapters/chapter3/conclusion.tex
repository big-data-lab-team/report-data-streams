\section{Conclusion}
In this chapter, we give a notation for original \acrshort{ltc} algorithm and a
pseudo-code with our notation and definitions. To generalize \acrshort{ltc} to
dimension $n$, the \acrshort{ltc} property has been re-written in
Section~\ref{sec:ltc-property}. According Section~\ref{sec:effect-norm}, we
known that the n-dimensional \acrshort{ltc} might get different results in
different norm. In Euclidean norm, it is more difficult to check the
intersection and record this inter section in memory. We introduce the method of
combination of plane sweep and bisection for intersection test in Euclidean
norm, but it spends more processing time and costs more memory to record a
sequence of n-balls than the method for the Infinity norm. In next chapter, we
will show the result of using n-dimensional \acrshort{ltc} on the accelerometer
data of human activities. 
