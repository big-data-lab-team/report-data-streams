\section{Conclusion}
In this chapter, we give a notation for the original \acrshort{ltc} algorithm
and a pseudo-code with our notation and definitions. To generalize
\acrshort{ltc} to dimension $n$, the \acrshort{ltc} property has been re-written
in Section~\ref{sec:ltc-property}. According to Section~\ref{sec:effect-norm},
we have known that the n-dimensional \acrshort{ltc} might get different results
in different norms. In the Euclidean norm, it is more difficult to check the
intersection and record this inter section in memory. We introduce the method of
combination of plane sweep and bisection for intersection test in the Euclidean
norm, but it spends more processing time and costs more memory to record a
sequence of n-balls than the method for the Infinity norm. In the next chapter,
we will show the result of using n-dimensional \acrshort{ltc} on the
accelerometer data of human activities. 
