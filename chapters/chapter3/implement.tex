\section{Implementation of LTC n-dimension}

To implement LTC in n dimensions with the infinity norm, we maintain a cuboid
representation of $\cap_{l=1}^j{\mathcal{B}_l}$ across the
iterations of the \texttt{while} loop in
Algorithm~\ref{algo:general-ltc}. The implementation works with
constant memory and requires limited CPU time.

% With the Euclidean norm, the intersection test is more complex. We keep in
% memory a growing set $S$ of n-balls and the bounding box $B$ of their
% intersection. 
With the Euclidean norm, the intersection test is more complex. We keep in
memory a growing set $S$ of n-balls and the bounding box $B$ which represents an
approximate range of their intersection. We define $B$ as follow:
\begin{equation*}
    B = \bigcap_{l=1}^{j-1} box(\mathcal{B}_l) 
\end{equation*}
Where \texttt{box()} is a function that returns the bounding box of an n-ball.
Then, when a new point arrives, we consider the associated n-ball
$\mathcal{B}_j$ and our intersection test works as in
Algorithm~\ref{algo:euclidean}.  Firstly, we check the intersection with
box($\mathcal{B}_j$) and $B$, which shows in Algorithm~\ref{algo:euclidean} line
9. Secondly, for each $\mathcal{B}_i$ belongs to $S$, check the intersection
between $\mathcal{B}_i$ and $\mathcal{B}_j$ (12 line in
Algorithm~\ref{algo:euclidean}). At the same time $\mathcal{B}_i$ will be
removed if it includes $\mathcal{B}_j$, because smaller n-ball intersects
$\mathcal{B}_j$ than a bigger n-ball which contains the smaller one(15 line in
Algorithm~\ref{algo:euclidean}).  Finally, we search a point in intersection of
all n-balls in $S$, using plane sweep~\ref{shamos1976geometric,
souvaine2008line} and bisection initialized by the bounds of B. Function
\texttt{find\_bisection(S, B)} ( see
Algorithm~\ref{algo:find_bisection_function}) and \texttt{recursive(S, L, R, N,
$X^n$)} (see Algorithm~\ref{algo:recursive}) show how it works. 

\TG{\#\#This paragraph has overlap with the previous one, writing should be revised
such that the description of the algorithm is straight and unambiguous.} 

% describe function find_bisection
In function \texttt{find\_bisection(S, B)}, we selected dimension of n-balls as
$n$ used at the beginning of bisection, also computed minimum/maximum value of
bounding box $B$ at the $n_{th}$ dimension, which corresponds to $left$ and
$right$ in bisection method. The return object $X^n$ represents a n-dimensional
point in the intersection of all n-balls in $S$. It would be returned if we have
$True$ result from function \texttt{recursive(S, L, R, N, $X^n$)}. $X^n$ is
initialized and put into function \texttt{recursive(S, L, R, N, $X^n$)}, because
it is necessary to update the values (see Algorithm~\ref{algo:recursive} line
28) of $X^n$ during the recursive process. In 30 line of
Algorithm~\ref{algo:recursive}, it happens in case of $left > right$, which
means:
\begin{equation*}
    mid \  \bigcap \  (min\_id \cap max\_id) = \O
\end{equation*}

In Algorithm~\ref{algo:euclidean} line 12, It guarantees any two n-balls in $S$
has intersection. Therefore, any point in the intersection of $min\_id$ and
$max\_id$, could be used for determining the direction of bisection. In our
implement, we select a point in connection of these two n-ball's center, which
also locates in intersecting object(line/chord when order=2, plane/circle when
order=3). Figure~\ref{fig:compare_point} shows the point we selected in 2D.

\TG{\#\#Algorithms are confusing: see my comments in the algorithms. Revise description accordingly.}\\

\begin{figure}
  \centering
\includegraphics[width=0.7\textwidth]{figures/point-in-chord.png}
\caption{The point selected to determine the direction}
\label{fig:compare_point}
\end{figure}


\TG{The chapter is missing a conclusion.}

\begin{algorithm}
\begin{algorithmic}[1]
\Input
    \Desc{$S$}{Set of intersecting n-balls}
    \Desc{$B$}{Bounding box of the intersection of n-balls in $S$}
    \Desc{$\mathcal{B}_j$}{New n-ball to check}
\EndInput
\Output
    \Desc{$S$}{Updated set of intersecting n-balls}
    \Desc{$B$}{Updated bounding box}
    \Desc{$T$}{True if all the n-balls in S and $\mathcal{B}_j$ intersect}
    \Desc{$x$}{Point inside intersection. It might be Null}
\EndOutput

%\If{$\mathcal{B}_j \cap B = \O$} \Comment{Ball is outside bounding box}
\If{box($\mathcal{B}_j$) $\cap B = \O$} \Comment{N-ball is outside bounding box}
    \State \Return ($S$, $B$, False, Null)
\EndIf
\If{$\exists\ \mathcal{B}_i \in S$ s.t. $\mathcal{B}_j \cap \mathcal{B}_i = \O$}
    \State \Return ($S$, $B$, False, Null) \Comment{Some n-balls don't intersect}
\EndIf
\If{$\exists\ \mathcal{B}_i \in S$ s.t. $\mathcal{B}_j \subset \mathcal{B}_i$} \Comment{Remove inclusions}
%\State Remove $\mathcal{B}_i$ from $B$. %Add $\mathcal{B}_j$ to $B$. 
    \State Remove $\mathcal{B}_i$ from $S$. %Add $\mathcal{B}_j$ to $S$. 
\EndIf
\State $B$ = box($\mathcal{B}_j$) $\ \bigcap\ $ $B$
\State $S$ = $S \ \bigcup \  \{\mathcal{B}_j\}$
\State $x$ = find\_bisection($S$, $B$) \Comment{This can take some time}
\If{$x$ == Null}
    \State \Return ($S$, $B$, False, Null)
\Else
    \State \Return ($S$, $B$, True, $x$)
\EndIf
\end{algorithmic}
\caption{Intersection test for Euclidean n-balls.}
\label{algo:euclidean}
\end{algorithm}


\begin{algorithm}
\begin{algorithmic}[1]
\Input
    \Desc{$S$}{Set of n-balls} \TG{\#\#I don't think you know that they are intersecting at this point. It should be candidate n-balls.}
    \BO{yes, you are right}
    \Desc{$B$}{Bounding box of the intersection of n-balls in $S$}
\EndInput
\Output
    \Desc{$X^n$}{N-dimensional point in all n-balls in $S$, $X^n = (x_1,...,x_n)$}
\EndOutput

\State $n$ = dimension of n-balls
\State $left$ = min\_value($B$, $n$) \Comment{Minimum value of $B$ along $n^{th}$ dimension}
\State $right$ = max\_value($B$, $n$) \Comment{Maximum value of $B$ along $n^{th}$ dimension}
\State $X^n$ = Null \TG{\#\#How is X initialized?}
\If{recursive(S, left, right, n, $X^n$)}
    \State \Return $X^n$
\Else
    \State \Return Null
\EndIf
\end{algorithmic}
\caption{Function find\_bisection(S, B)}
\label{algo:find_bisection_function}
\end{algorithm}

\begin{algorithm}
\begin{algorithmic}[1]
\Input
   \Desc{$S$}{Set of n-balls} \TG{\#\#There's no guarantee that they intersect when this function is called.}
   \Desc{$L$}{Min value used in bisection}
   \Desc{$R$}{Max value used in bisection}
   \Desc{$N$}{The $N^{th}$ dimension, $N \in \{1...n\}$}
   \TG{??Why not just use $n$?}
   \Desc{$X^n$}{N-dimensional point which records point of intersection, $X^n$ = $(x_1,...,x_n)$} \TG{\#\#If you know the point already, what are you looking for?}
\EndInput
\Output
   \Desc{$T$}{True if a point is found by using bisection}
\EndOutput
\If{$N$ == 1}
    \State $X^n.x_N$ = $\frac{L+R}{2}$ \TG{\#\#isn't it $X^n$.$x_N$?} 
    \State \Return True
\EndIf
\While{$L \leqslant R$ }
    \State $mid$ = $\frac{L+R}{2}$
    \State $surface$ = $\{x_{N}...x_{n}\}$
    \State $left$ = -$\infty$, $\ right$ = +$\infty$
    \ForAll{$\mathcal{B}_i \in S$}
        \State $min$ = min\_value($\mathcal{B}_i \cap surface$, N-1)
        \State $max$ = max\_value($\mathcal{B}_i \cap surface$, N-1)
        \If{$left < min$}
            \State $left$ = $min$, $\ min\_id$ = $\mathcal{B}_i$
        \EndIf
        \If{$right > max$}
            \State $right$ = $max$, $\ max\_id$ = $\mathcal{B}_i$
        \EndIf
    \EndFor
    
    \If{$left \leqslant right$}
        \State $X^n.x_N$ = $mid$
        \State \Return recursive($S$, $left$, $right$, $N-1$, $X^n$)
    \ElsIf{$\ \forall\ P$ (Point) $\in (min\_id \cap max\_id)$ s.t. $P.x_N < mid$}
        \State $right$ = $mid$
    \Else
        \State $left$ = $mid$
    \EndIf
\EndWhile
\State \Return False
\end{algorithmic}
\caption{Function recursive(S, L, R, N, $X^n$)}
\label{algo:recursive}
\end{algorithm}
