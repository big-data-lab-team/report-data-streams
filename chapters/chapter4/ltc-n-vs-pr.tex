\section{LTC n-dimension compare with Regression}

From the previous experiments, we know that LTC has good performance on  walking
and running accelerometer data sets. However as we seen, the running data set
(Figure~\ref{fig:datasets-2}) looks like it is made of many high-degree plane
curves. We wonder whether high-degree Regression compression method performs
better with our accelerometer data sets. In this paragraph, we implement the
Polynomial Regression compression method mentioned in~\ref{sec:polynomial} with
LTC in different dimensions.

% need  a graph for polynomial regression with different p,  by using 200Hz
% walking and running data set, also need a table to show the RMSE and
% compression ratio, Max error.

\subsection{Implementation of Polynomial Regression}

We implemented polynomial regression method to compress 3D acceleration data
($t$, $a.x$, $a.y$, $a.z$). In the implementation, we process polynomial
regression compression method on each accelerometer parameter, with the
relationship between independent variables $t$ (Time-stamp) and dependent
variables $a.x$, $a.y$ or $a.z$ (e.g. $t \sim a.x$). It means  three regression
models shell be transmitted to represent a curve in 3D space. For tolerance
checking, we use infinity norm and Euclidean norm.

\todo{re-write section~\ref{sec:polynomial} according to follow describe}
As we mentioned in Section~\ref{sec:polynomial}, $M_{j}^{k}$ means the $j^{th}$
polynomial regression model after $k^{th}$ transmission, but to apply this
method on the acceleration data which is 3-dimensional, we need three model for
it (each for per parameter). Assuming $Mx_{j}^{k}$, $My_{j}^{k}$ and
$Mz_{j}^{k}$ represent the regression model on each parameter $x$, $y$ and $z$
respectively. Algorithm~\ref{algo:n-polynomial} shows how polynomial regression
is applied on 3D acceleration data. In line 13, the coefficients of three
regression model of parameters $x$, $y$ and $z$ are calculated. The function
\texttt{max\_residue()} returns the maximum error between predicted data and
original data. If the maximum residue is larger than $\epsilon$, then the last
regression models which meet error tolerance and the time-stamp of last data are
transmitted as compression result.
\begin{algorithm}
\begin{algorithmic}[1]
\Input
    \Desc{($\chi, t_i$)}{$\quad $Received data stream at time $t_i$}
    \Desc{$\epsilon$}{$\quad $Error bound}
    \Desc{$p$}{$\quad $The order of polynomial function}
\EndInput
\Output
    \Desc{tr}{Transmitted coefficients}
\EndOutput
\State $S$ = $\O$
\State $k$=0; $j$=0

\While{True}
    \State $S = S \cup (\chi, t_i)$
    \If{$S.length() \geqslant p+1$}
        \State j += 1
        \State ($Mx_{j}^{k}$, $My_{j}^{k}$, $Mz_{j}^{k}$) = model($S$, $p$)    \Comment{Compute coefficients}
        \If{max\_residue($Mx_{j}^{k}$, $My_{j}^{k}$, $Mz_{j}^{k}$, $S$) $> \epsilon$}
            \State tr = ($Mx_{j-1}^{k}$, $My_{j-1}^{k}$, $Mz_{j-1}^{k}$, $t_{i-1}$)
            \State k += 1; j = 0
            \State $S$ = $\chi$
        \EndIf
    \EndIf
\EndWhile

\end{algorithmic}
\caption{Polynomial Regression Algorithm for 3D Accelerometer data}
\label{algo:n-polynomial}
\end{algorithm}

We use implemented polynomial regression method on previous walking and running
data set. Figure~\ref{fig:poly-regression-3-degree} and
Figure~\ref{fig:poly-regression-5-degree} shows the reconstructed data. In the
compression process, we need to keep in memory all the data between two
transmissions and update regression models when each new data arriving. So,
using polynomial regression method needs more memory and processing time than
n-dimensional LTC for Infinity norm, but for Euclidean norm, we need to also
record n-balls between two transmissions and execute intersection checking. 

\begin{figure*}
\centering
\begin{subfigure}{\columnwidth}
\centering
\includegraphics[width=0.3\columnwidth]{figures/regression-p3-in-x.pdf}
\includegraphics[width=0.3\columnwidth]{figures/regression-p3-in-y.pdf}
\includegraphics[width=0.3\columnwidth]{figures/regression-p3-in-z.pdf}
\caption{Regression degree=3 with Infinity norm}
\end{subfigure}

\centering
\begin{subfigure}{\columnwidth}
\centering
\includegraphics[width=0.3\columnwidth]{figures/regression-p3-Eu-x.pdf}
\includegraphics[width=0.3\columnwidth]{figures/regression-p3-Eu-y.pdf}
\includegraphics[width=0.3\columnwidth]{figures/regression-p3-Eu-z.pdf}
\caption{Regression degree=3 with Euclidean norm}
\end{subfigure}

\caption{Reconstructed data by using 5-degree Regression compression method}
\label{fig:poly-regression-3-degree}
\end{figure*}

%-------------------------------------------------------%

\begin{figure*}
\centering
\begin{subfigure}{\columnwidth}
\centering
\includegraphics[width=0.3\columnwidth]{figures/regression-p5-in-x.pdf}
\includegraphics[width=0.3\columnwidth]{figures/regression-p5-in-y.pdf}
\includegraphics[width=0.3\columnwidth]{figures/regression-p5-in-z.pdf}
\caption{Regression degree=5 with Infinity norm}
\end{subfigure}

\centering
\begin{subfigure}{\columnwidth}
\centering
\includegraphics[width=0.3\columnwidth]{figures/regression-p5-Eu-x.pdf}
\includegraphics[width=0.3\columnwidth]{figures/regression-p5-Eu-y.pdf}
\includegraphics[width=0.3\columnwidth]{figures/regression-p5-Eu-z.pdf}
\caption{Regression degree=5 with Euclidean norm}
\end{subfigure}

\caption{Reconstructed data by using 5-degree Regression compression method}
\label{fig:poly-regression-5-degree}
\end{figure*}

\subsection{Results of Comparison}
In our implementation, for a regression model which has N-degree and
M-parameters we need M*(N+1) coefficients and extra time-stamp used to record
where should the model stop. In the Table , It obvious that LTC n-dimension work
better than Polynomial regression, because it need more bytes to save the
coefficients. In our case, coefficients are recorded as float points which would
use 4 bytes per each. The total number of bytes needed to represent original
data is (N+1)*M*4+4 (Int32 for Time-stamp). Polynomial regression using more
processing time to calculate coefficients and build model when each new data
point comes. 

From the Table~\ref{table:CR-compare-walking} and Table
~\ref{table:CR-compare-running}, We can learn that increasing the Dimension $n$
is detrimental to compression ratio with both n-dimensional LTC and Polynomial
Regression, and LTC gives better compression ratio then regression method on
different dimensions. Also same with experiment 2 in
Section~\ref{sec:experiment2-ltc}, the Compression ratio is higher for the
infinity norm than for the Euclidean and is higher for the walking than for
running. For given degree of polynomial regression method, the higher degree
results smaller compression ratio for walking and running in general, but the
variations are slight. However, in Table~\ref{table:CR-compare-walking}, when
dimension is 2 and 3, the 5-degree regression regression compresses more data
then 3-degree regression method by using Infinity norm to check error tolerance
in walking data set, and in Table~\ref{table:CR-compare-running} the 5-degree
regression method works better when dimension equals 1 for both norm, equals 2
for infinity norm.



\begin{table}[]
\begin{tabular}{|l|l|l|l|l|l|l|l|}
\hline
           & \multicolumn{2}{c|}{LTC} & \multicolumn{2}{c|}{Regression degree = 3} & \multicolumn{3}{l|}{Regression degree = 5}\\ \hline
Dimension & Infinity   & Euclidean   & Infinity    & Euclidean     & Infinity      & \multicolumn{2}{l|}{Euclidean} \\ \hline
1          & 90.1\%     & 90.1\%      & 89\%        & 89\%          & 88.96\%       & \multicolumn{2}{l|}{88.96\%}   \\ \hline
2          & 89.6\%     & 88.7\%      & 83.35\%     & 83.35\%       & 83.55\%       & \multicolumn{2}{l|}{83.1\%}    \\ \hline
3          & 88.9\%     & 87.6\%      & 80.24\%     & 78.68\%       & 80.8\%        & \multicolumn{2}{l|}{77.96\%}   \\ \hline
\end{tabular}
\caption{Comparisons of compression ratio on Walking data set}
\label{table:CR-compare-walking}
\end{table}


\begin{table}[]
\begin{tabular}{|l|l|l|l|l|l|l|l|}
\hline
           & \multicolumn{2}{c|}{LTC} & \multicolumn{2}{c|}{Regression degree = 3} & \multicolumn{3}{l|}{Regression degree = 5}          \\ \hline
Dimension & Infinity   & Euclidean   & Infinity            & Euclidean           & Infinity& \multicolumn{2}{l|}{Euclidean} \\ \hline
1          & 74.7\%     & 74.7\%      & 70.7\%      & 70.7\%        & 71.1\%        & \multicolumn{2}{l|}{71.1\%}    \\ \hline
2          & 70.6\%     & 68.6\%      & 58.3\%      & 57.4\%        & 58.4\%        & \multicolumn{2}{l|}{57.35\%}   \\ \hline
3          & 68.6\%     & 64.4\%      & 51.1\%      & 48\%          & 50.2\%        & \multicolumn{2}{l|}{47.92\%}   \\ \hline
\end{tabular}
\caption{Comparisons of compression ratio on Running data set}
\label{table:CR-compare-running}
\end{table}
