\section{Evaluation of LTC n-dimension}
%\section{LTC n-dimension with LTC original}
\label{sec:ltc-n-dimensions}

We conducted two experiments using Motsai's Neblina
module, a system
with a Nordic Semiconductor nRF52832 micro-controller, 64~KB of RAM,
and Bluetooth Low Energy connectivity. Neblina has a 3D
accelerometer, a 3D gyroscope, a 3D magnetometer, and environmental
sensors for humidity, temperature and pressure. The platform is
equipped with sensor fusion algorithms for 3D orientation tracking and
a machine learning engine for complex motion analysis and motion
pattern recognition~\cite{sarbishei2016accuracy}. Neblina has a
battery of 100mAh; at 200~Hz, its average consumption is 2.52~mA when using
accelerometer and gyroscope sensors but without radio
transmission, and 3.47~mA with radio transmission, leading to an
autonomy of 39.7~h without transmission and 28.8~h with transmission.

\subsection{Experiment 1: validation}

We validated the behavior of our LTC extension on a PC using data
acquired with Neblina. We collected two 3D accelerometer time-series, a
short one and a longer one, acquired on two different subjects
performing biceps curl, with a 50~Hz sampling rate (see
Figure~\ref{fig:datasets-1}). In both cases, the subject was wearing
Neblina on their wrist. It should be noted that the longest
time-series also has a higher amplitude, perhaps due to differences between
subjects.

\begin{figure*}
\centering
\begin{subfigure}{\columnwidth}
\centering
\includegraphics[width=0.3\columnwidth]{figures/5-time-bicep-curl-plot-x.pdf}
\includegraphics[width=0.3\columnwidth]{figures/5-time-bicep-curl-plot-y.pdf}
\includegraphics[width=0.3\columnwidth]{figures/5-time-bicep-curl-plot-z.pdf}
\caption{Short biceps curl$^a$}
\end{subfigure}

{\footnotesize $^a$ Average
of az data is -7.81mg. It was shifted to 0 so that the graphs can all
use the same y scale.}

\begin{subfigure}{\columnwidth}
\centering
\includegraphics[width=0.3\columnwidth]{figures/Mohammad-bicep-curl-plot-x.pdf}
\includegraphics[width=0.3\columnwidth]{figures/Mohammad-bicep-curl-plot-y.pdf}
\includegraphics[width=0.3\columnwidth]{figures/Mohammad-bicep-curl-plot-z.pdf}
\caption{Long biceps curl}
\end{subfigure}

\caption{Time-series used in Experiment 1}
\label{fig:datasets-1}
\end{figure*}

We compressed the time-series with various values of $\epsilon$, using
our 2D (x and y) and 3D (x, y and z) implementations of LTC. On
Neblina, the raw uncalibrated accelerometer data corresponds to errors
around 20~mg (1~g is 9.8~m/s$^2$). We used a
laptop computer with 16~GB of RAM, an Intel i5-3210M CPU @ 2.50GHz
$\times$ 4, and Linux Fedora 27. We measured memory consumption using
Valgrind's massif
tool~\cite{nethercote2006building},
and processing time using \texttt{gettimeofday()} from the GNU C
Library.

Results are reported in Table~\ref{table:results-validation}.
As expected, the compression ratio increases with $\epsilon$, and the
maximum measured error remains lower than $\epsilon$ in all cases. The
maximum is reached most of the time on these time-series.

\paragraph{Infinity vs Euclidean norms}
The average ratio between the compression ratios obtained with the infinity and
Euclidean norms is 1.03 for 2D data, and 1.06 for 3D data. These ratios are
lower than the theoretical values of $\frac{4}{\pi}$ in 2D and $\frac{6}{\pi}$
in 3D, which are obtained for random-uniform signals. Unsurprisingly, the
infinity norm surpasses the Euclidean norm in terms of resource consumption.
Memory-wise, the infinity norm requires a constant amount of 80~B, used to store
the intersection of n-balls. The Euclidean norm, however, uses up to 4.7~KB of
memory for the Long time-series in 3D with $\epsilon$=48.8~mg. More importantly,
the amount of required memory increases for longer time-series, and it also
increases with larger values of $\epsilon$. Similar observations are made for
the processing time, with values ranging from 0.4~ms for the simplest
time-series and smallest $\epsilon$, to 41.3~ms for the most complex time-series
and largest~$\epsilon$.
%~ Figure~\ref{fig:memory} shows the memory consumption
%~ of the 3D Euclidean implementation for $\epsilon$=48.8~mg: peaks appear
%~ at the end of long compressed ranges where the signal was $\epsilon$-closed
%~ to the linear approximation.

\paragraph{2D vs 3D}
For a given $\epsilon$, the compression
ratios are always higher in 2D than in 3D. It makes sense since the
probability for the signal to deviate from a straight line
approximation is higher in 3D than it is in 2D. Besides, resource
consumption is higher in 3D than in 2D: for the infinity norm, 3D
consumes 1.4 times more memory than 2D (1.8 times on average for
Euclidean norm), and the processing time is 1.35 longer (1.34 on
average for Euclidean norm).

\begin{table}
    \begin{subfigure}{\columnwidth}
    \centering
    \begin{tabular}{l|l|l|l|l}
    \hline
    \rowcolor{headcolor}
                           & \multicolumn{2}{c|}{Infinity} & \multicolumn{2}{c}{Euclidean}\\
    \rowcolor{headcolor}
    $\epsilon$  (mg)          & 48.8         & 34.5       & 48.8       & 34.5 \\
    \hline
    Max error   (mg)          & 48.8         & 34.4       & 48.8       & 34.5 \\
    Compression ratio (\%)    & 79.77        & 72.59      & 77.49      & 70.96\\
    Peak memory   (B)         & 80           & 80         & 688        & 688  \\
    Processing time (ms)      & 0.101        & 0.094      & 0.456      & 0.406\\ \hline
    \end{tabular}
    \caption{Short biceps curl (2D)}
    \end{subfigure}\\
    \begin{subfigure}{\columnwidth}
    \centering
    \begin{tabular}{l|l|l|l|l}
    \hline
    \rowcolor{headcolor}
                   & \multicolumn{2}{c|}{Infinity} & \multicolumn{2}{c}{Euclidean} \\
    \rowcolor{headcolor}
    $\epsilon$ (mg)            & 48.8        & 34.5       & 48.8        & 34.5    \\
    \hline
    Max error  (mg)            & 48.8        & 34.5       & 48.8        & 34.5             \\
    Compression ratio (\%)     & 77.46       & 70.98      & 75.77       & 68.81           \\
    Peak memory  (B)           & 80          & 80         & 2512        & 2608             \\
    Processing time (ms)       & 6.06        & 5.84       & 33.84       & 31.07           \\ \hline
    \end{tabular}
    \caption{Long biceps curl (2D)}
    \end{subfigure}\\
    \begin{subfigure}{\columnwidth}
    \centering
    \begin{tabular}{l|l|l|l|l}
    \hline
    \rowcolor{headcolor}
                           & \multicolumn{2}{c|}{Infinity} & \multicolumn{2}{c}{Euclidean} \\
    \rowcolor{headcolor}
    $\epsilon$ (mg)        & 48.8          & 28.2          & 48.8   & 28.2   \\
    \hline
    Max error  (mg)        & 48.8          & 28.2          & 48.8   & 28.2   \\
    Compression ratio (\%) & 78.14         & 66.39         & 74.39  & 63.13   \\
    Peak memory (B)        & 112           & 112           & 1744   & 784    \\
    Processing time (ms)   & 0.147         & 0.134         & 0.731  & 0.514  \\ \hline
    \end{tabular}
    \caption{Short biceps curl (3D)}
    \end{subfigure}\\
    \begin{subfigure}{\columnwidth}
    \centering
    \begin{tabular}{l|l|l|l|l}
    \hline
    \rowcolor{headcolor}
                              & \multicolumn{2}{c|}{Infinity} & \multicolumn{2}{c}{Euclidean} \\
    \rowcolor{headcolor}
    $\epsilon$ (mg)                & 48.8        & 28.2       & 48.8     & 28.2    \\
    \hline
    Max error  (mg)                & 48.8        & 28.2       & 48.8     & 28.2    \\
    Compression ratio (\%)         & 71.23       & 58.11      & 67.35    & 53.24   \\
    Peak memory (B)                & 112         & 112        & 4752     & 3856    \\
    Processing time (ms)           & 7.87        & 7.22       & 41.29    & 39.04   \\ \hline
    \end{tabular}
    \caption{Long biceps curl (3D)}
    \label{tabl:results-validation-d}
    \end{subfigure}
    \caption{Results from Experiment 1}
    \label{table:results-validation}
\end{table}

\begin{figure*}
\centering
\begin{subfigure}{\columnwidth}
\centering
\includegraphics[width=0.3\columnwidth]{./figures/walking-1000-x.pdf}
\includegraphics[width=0.3\columnwidth]{./figures/walking-1000-y.pdf}
\includegraphics[width=0.3\columnwidth]{./figures/walking-1000-z.pdf}
\caption{Walking}
\end{subfigure}
\begin{subfigure}{\columnwidth}
\centering
\includegraphics[width=0.3\columnwidth]{./figures/running-1000-x.pdf}
\includegraphics[width=0.3\columnwidth]{./figures/running-1000-y.pdf}
\includegraphics[width=0.3\columnwidth]{./figures/running-1000-z.pdf}
\caption{Running}
\end{subfigure}
\caption{Time series used in Experiment 2 \vspace*{-0.3cm}}
\label{fig:datasets-2}
\end{figure*}

%~ \begin{figure}
%~ \includegraphics[width=\columnwidth]{./figures/memory-mohammad.pdf}
%~ \caption{Memory usage of 3D Euclidean implementation, long biceps curl time-series, $\epsilon$=48.8~mg.}
%~ \label{fig:memory}
%~ \end{figure}

\subsection{Experiment 2: impact on energy consumption}
\label{sec:experiment2-ltc}
We acquired two 3D accelerometer time-series at 200~Hz for two
activities: walking and running (see Figure~\ref{fig:datasets-2}). In
both cases, the subject was wearing Neblina on their wrist as in
Experiment 1. We collected 1,000 data points for each activity,
corresponding to 5 seconds of activity.

We measured energy consumption associated with the transmission of
these time-series by ``replaying" the time-series after loading them as
a byte array in Neblina. We measured the current every 500~ms. We also
measured the max and average latency resulting from compression.

Results are reported in Table~\ref{table:results-energy}. For a given
$\epsilon$ and norm, the compression ratio is larger for walking than
for running. The ratio of saved energy is relative to the reference
current of 3.47~mA measured when Neblina transmits data without
compression. In all cases, activating compression saves energy. The
reduction in energy consumption behaves as the compression ratio: it
increases with $\epsilon$, it is higher for the infinity norm than for
the Euclidean, and it is higher for the walking activity than for
running. For a realistic error of $\epsilon$=9.8~mg, the ratio of
saved energy with the infinity norm is close to 20\% for the walking
activity, which is substantial. Latency is higher for walking
than for running, and it is also higher for the Euclidean norm than
for the infinity norm. In all cases, the latency remains lower
than the 5-ms tolerable latency at 200~Hz, which demonstrates the
feasibility of 3D LTC compression.

\begin{table}
   \begin{subfigure}{\columnwidth}
   \centering
   \begin{tabular}{l|l|l|l|l|l|l}
   \hline
   \rowcolor{headcolor}
                          & \multicolumn{3}{c|}{Infinity} & \multicolumn{3}{c}{Euclidean} \\
   \rowcolor{headcolor}
   $\epsilon$ (mg)             & 48.8    & 9.8      & 4.9   & 48.8   & 9.8   & 4.9   \\
   \hline
   Max error (mg)              & 48.8    & 9.8      & 4.9   & 48.8   & 9.8   & 4.9   \\
   Compr.      ratio (\%)      & 88.9    & 66.4     & 45.5  & 87.6   & 63.3  & 37.2  \\
   Average (mA)                & 2.64    & 2.79     & 3.02  & 3.10   & 3.02  & 3.13  \\
   Saved energy (\%)           & 23.9    & 19.7     & 13.0  &  10.7  & 13.0  & 9.7   \\
   Max latency ($\mu$s)& 60      & --       & --    & 1530   & --    & --    \\
   Average latency ($\mu$s) & 31 & --       & --    & 145    & --    & --    \\
   \hline
   \end{tabular}
   \caption{Walking}
   \end{subfigure}
   \begin{subfigure}{\columnwidth}
   \centering
   \begin{tabular}{l|l|l|l|l|l|l}
   \hline
   \rowcolor{headcolor}

                     & \multicolumn{3}{c|}{Infinity} & \multicolumn{3}{c}{Euclidean} \\
    \rowcolor{headcolor}
   $\epsilon$ (mg)        & 48.8       & 9.8      & 4.9       & 48.8      & 9.8    & 4.9    \\
   \hline
   Max error (mg)         & 48.8       & 9.8      & 4.9       & 48.8      & 9.8    & 4.9   \\
   Compr.      ratio (\%) & 68.6       & 25.5     & 9.5       & 64.4      & 19.8   & 5.7   \\
   Average (mA)        & 2.88     & 3.22   & 3.38     & 2.95    & 3.32    & 3.39   \\
   Saved energy (\%)      & 17.0    & 7.2    & 2.5   & 14.9   & 4.3   & 2.2\\
   Max latency ($\mu$s)& 60      & --       & --    & 840   & --    & --    \\
   Average latency ($\mu$s) & 30 & --       & --    & 64    & --    & --    \\
   \hline
   \end{tabular}
   \caption{Running}
   \end{subfigure}
   \caption{Results from Experiment 2}
   \label{table:results-energy}
\end{table}
