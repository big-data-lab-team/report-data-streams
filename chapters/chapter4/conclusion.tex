\section{Conclusion}
This chapter demonstrated the effect of n-dimensional LTC. In the first section,
we compressed biceps curl, walking and running steam data in different
dimensions by using n-dimensional LTC. With error bound $\epsilon$=48.8\ mg,
3-dimensional LTC can compress at least 67\% of original data stream.
In section~\ref{sec:experiment2-ltc}, we deployed LTC n-dimension on Neblina to
measure the impact on energy consumption. We processed the experiment on Neblina
with 3D accelerometer streams for walking and running activities. According to
the Table~\ref{table:table:results-energy}, LTC n-dimension can reduce data
transmission up to 88.9\% , and help Neblina to save energy up to 23.9\%, where
the error bound $\epsilon$ = 48.8 mg. 
These experiments and results show that n-dimensional LTC is feasible on
connected objects and it has good performance for compressing stream data to
reduce energy consumption of transmission in IoT networks. 
