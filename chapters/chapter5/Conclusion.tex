\chapter{Conclusion}

\todo{explain what has been implemented by Motsai On Neblina, in Conclusion
chapter}


\section{Summary}

.... introduction...
... LTC ...
... no method for multi-dimension...

In this thesis, we extend LTC compression algorithm to $n$ dimensions and give its formulation. This extension method is norm-independent and can work with the data stream have $n$ attributes. 


%   \item Formalize the description of original LTC algorithm
%   \item Extend LTC to dimension $n$.
%   \item Propose an algebraic formulation of n-dimensional LTC algorithm.
%   \item Introduce an norm-independent expression of n-dimensional LTC, according
%   to the algebraic formulation.
we extend LTC to n dimension, and 
formalizes the description initially proposed
in~\cite{schoellhammer2004lightweight} and presents our norm-independent
extension to dimension $n$, together with description of our implementation in

extension to n dimension and give his performance in different norm Euclidean and Manhattan  infinity norm
give implement and test the result through Neblina, compare the compression ratio with different dimension and different norm
and compare the energy saving

Compare with the Polynomial regression with has best compression ratio in ~\cite{zordan2014performance}.

how every we did measure the energy consumption, just give compression ratio




\section{Limitation}

In LTC with n-dimension the main limitation is the memory usage, in Euclidean norm, with can be considered as the intersection of n-ball.
It is difficult to compute this. we found that the theory .......
How every it takes ... time complexity
We provide a method .... which give $O(n\times (\log{n})^D)$

It is also too complexity and spend time to short interval communication.(the last data point has not been process completely yet when new data coming)
according to the result in Chapter 4, the method we provide can work for the transmission rate in 200Hz, but if the the interval increase, it might require more memory.

% 我们希望选择一个 方法 不会依赖数据的重复性, 如果在模写情况下,这个数据成一定 固定的模式
% 那么 其他model可能会提供更多的compression ratio
% 比如说 LEC S-LEC 字典型压缩算法  会对那些  出现重复几率大,或者重复模式时候 提供更好的效果。

\section{Future work}

尝试 其他算法, 是否可以提高 energy saving, 尝试寻找 更小复杂度 时间空间都是 解决n-ball 的intersection problem
sometimes, we wanna lossless data, so a frame work which can support a efficent lossless algorithem and lossy algorithm is need. 
从 s-lec 来看, it could give good compression ratio than other lossless method.
try to find a method which can support multi-dimension or extend it into n-dimension

check the Slides of before meeting 
