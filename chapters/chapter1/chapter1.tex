\chapter{Introduction}

\TG{At the end of the introduction, you should mention that the work
described in Ch3 and 4 was published and give the reference.}

\TG{Mention OrpaillCC (you are a co-author). Mention internship and
implementation at Motsai.}

%%----------------------------------------------------------------------------%%
%IoT, Embedded system, sensors,
%Motsai(internship coding), Neblina, Big Data, Stream

% context
% problem description
% goal motivation
% outline and contribution(publication)
% sensor network 技术将会引来更多的挑战,随着无线连通的普及,the units in IoT will reach 26 billion by 2020(Gartnar 2014)
% IoT越来越普及,因为智能家居的普及,我们的生活就围绕IoT中。再这个网络中objects 可以互相交流。很多家居都拥有嵌入式系统,从而可以通过终端e.g.手机,电脑进行控制,比如用手机操作灯的开 关,空调温度的高低...同时多样的传感器时生活更加智能,aceleramiter....
% 给我们生活提供更加有效的信息。同时当big data, 数据分析兴起的时候,数据变得尤为重要,越来越有价值,但是因为sensor 产生的都是stream data, 它带来了一些挑战。
% 1,2,3
% (数据流无法再次查询,需要空间存储数据,但是devices空间是由限制的,尽管)
% 因此再数据流信息的挖掘中,有很多之前的研究。
% 。。。。
% 。。。
% 。。。。
% 然而再现实世界中,sensor 网络中的devices没有过多的空间 severs KB, 并且电量是有限的。因此在sensor网络中,如果我们要传输大量的数据,电量会消耗很多。数据需要发发送到client端,不管是通过 Wifi, Bluetooth etc. 

%%----------------------------------------------------------------------------%%

%IoT embedded system
With the recent technological advances internet of Things(IoT) applications, the
connected devices In IoT will reach 75 billion by 2025
\footnote{\url{https://www.statista.com/statistics/471264/iot-number-of-connected-devices-worldwide}}. 
So far the IoT has been involved in many field e.g. medical care, military,
sports and industrial manufacturing~\cite{boudargham2017exhaustive,
lai2013survey, da2014internet}. In industrial domains, the connected objects are
often used for capturing properties such as temperature, and receiving signal or
data from others. In domestic domains, many household products are expected to
provide more function which could make human activities more convenient and
improve the quality of life. For example, people can remotely control household
equipment with our smart electronics e.g. mobile phone and PC, through
aggregating embedded system with those equipment. Embedded system plays an
important role in IoT because it can increase processing power of devices and
execute specific task or application. 

The most typical network of connected object is WSNs, which are being
increasingly deployed for enabling continuous monitoring and sensing of physical
variables of our world~\cite{li2016temporal}.

And with the rise of data science, the data become more and more important cause
it could provide knowledge and information after filter and learning. Thus The
stream data produced in IoT also important. But cause the characteristic of
stream data, it it needed to process stream data timely due to its volume and
velocity. We may lose the opportunity to process them at all, if we did not do
it in real time. So some previous research has discussed various stream analytic
problems, such as estimating cardinality, frequency moments and clustering.

Power consumption is among the biggest challenges targeting connected 
objects, particularly in the industrial domains, where several sensing 
systems are commonly launched in the field to run for days or even 
weeks without being recharged. Typically, such devices use sensors to 
capture properties such as temperature or motion, and stream them to a 
host system over a radio transmission protocol such as Bluetooth 
Low-Energy (BLE). System designers aim to reduce the rate of data 
transmission as much as possible, as radio transmission is a 
power-hungry operation.

Compression is a key technique to reduce the rate of radio 
transmission.  While in several applications lossless compression 
methods are more desirable than lossy compression techniques, in the 
context of IoT and sensor data streams, the measured sensor data 
intrinsically involves noise and measurement errors, which can 
be treated as a configurable tolerance for a lossy compression algorithm. 

Resource-intensive lossy compression algorithms such as the ones based on 
polynomial interpolation, discrete cosine and Fourier transforms, or 
auto-regression methods~\cite{lu2010optimized} are not well-suited for 
connected objects, due to the limited memory available on 
these systems (typically a few KB), and the energy consumption 
associated with CPU usage. Instead, compression algorithms need 
to find a trade-off between reducing network communications and 
increasing memory and CPU usage. As 
discussed in~\cite{zordan2014performance}, linear compression methods 
provide a very good compromise between these two factors, leading to 
substantial energy reduction.
