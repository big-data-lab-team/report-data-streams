\chapter{Introduction}

\TG{At the end of the introduction, you should mention that the work
described in Ch3 and 4 was published and give the reference.}

\TG{Mention OrpaillCC (you are a co-author). Mention internship and
implementation at Motsai.}

%------------------------------------------------------------------------------%
%IoT, Embedded system, sensors,
%Motsai(internship coding), Neblina, Big Data, Stream

% context
% problem description
% goal motivation
% outline and contribution(publication)

% sensor network 技术将会引来更多的挑战,随着无线连通的普及,the units in IoT will reach 26 billion by 2020(Gartnar 2014)
% IoT越来越普及,因为智能家居的普及,我们的生活就围绕IoT中。再这个网络中objects 可以互相交流。很多家居都拥有嵌入式系统,从而可以通过终端e.g.手机,电脑进行控制,比如用手机操作灯的开 关,空调温度的高低...同时多样的传感器时生活更加智能,aceleramiter....
% 给我们生活提供更加有效的信息。同时当big data, 数据分析兴起的时候,数据变得尤为重要,越来越有价值,但是因为sensor 产生的都是streaming data, 它带来了一些挑战。
% 1,2,3
% (数据流无法再次查询,需要空间存储数据,但是devices空间是由限制的,尽管)
% 因此再数据流信息的挖掘中,有很多之前的研究。
% 。。。。
% 。。。
% 。。。。
% 然而再现实世界中,sensor 网络中的devices没有过多的空间 severs KB, 并且电量是有限的。因此在sensor网络中,如果我们要传输大量的数据,电量会消耗很多。数据需要发发送到client端,不管是通过 Wifi, Bluetooth etc. 

% 
% %Iot
% 随着IoT 的普及, 2025.....
% 到目前位置 IoT 被使用到很多领域
% 
% % sensors and WSN
% WSN 是IoT 的一种,并被普遍使用在工业中
% sensors 通过有限,或者low energy 的无线技术 进行连接,很多数据通过sensor measure 并传输到 sink sensor 或者 client 端
% 
% % data stream, big data
% 在 IoT 网络中传输的数据 是 data stream, 他与普通数据的不同是 。。。。。
% 随着数据科学的兴起,数据变得有位的重要,而 objected devices 产生的数据同样具有可挖掘的价值。但是因为streaming data 的特性 我们需要处理他in real-time 否则可能会失去它。
% 
% % Embedded system
% 
% 嵌入式系统 提供给 connected objects 计算能力,使得 devices 有一定的计算能力,从而我们可以进行一些操作,
% 比如说 online analysis (也可以offline)
% 
% % Problem
% however 电力消耗是一个 很大的问题,因为很多devices 是wireless 的 需要连续工作很长时间才会充电
% 
% % Neblina
% Neblina 是一个......

% goal 

% outline

%-----------------------------------------------------------------------------%

% -------------------- context------------------------%

%% -------IoT embedded system--------%%
With the recent technological advances internet of Things(IoT) applications, the
connected devices In IoT will reach 75 billion by 2025
\footnote{\url{https://www.statista.com/statistics/471264/iot-number-of-connected-devices-worldwide}}. 
So far the IoT has been involved in many field e.g. medical care, military,
sports and industrial manufacturing~\cite{boudargham2017exhaustive,
lai2013survey, da2014internet}. In industrial domains, the connected objects are
often used for capturing properties such as temperature, and receiving signal or
data from others. In domestic domains, many household products are expected to
provide more function which could make human activities more convenient and
improve the quality of life. For example, people can remotely control household
equipment with our smart electronics e.g. mobile phone and PC, through
aggregating embedded system with those equipment. 

%% -----streaming data, big data ----%%
The most typical network of connected object is WSNs, which are being
increasingly deployed for enabling continuous monitoring and sensing of physical
variables of our world~\cite{li2016temporal}. Sensor devices are connected by
network cables or low energy wireless network e.g. \texttt{Bluetooth Low Energy}
(BLE)\footnote{\url{https://web.archive.org/web/20170310111443/https://www.bluetooth.com/what-is-bluetooth-technology/how-it-works/low-energy}}
and \texttt{IEEE 802.15.4}\footnote{\url{http://www.ieee802.org/15/pub/TG4.html}}
and send the streaming data to sink sensor node or client side.

%% -----data stream, big data--------%%
\todo{write the part of big data}

\todo{add reference}
\BO{what i'mma write}

Streaming data is produced by connected objects in IoT and transferred over the
network.  And with the rise of data science, the data become more and more
important cause it could provide knowledge and information after filter and
learning. Thus The streaming data produced in IoT also important. But cause the
characteristic of streaming data, it is needed to process streaming data timely
due to its volume and velocity. We may lose the opportunity to process them at
all, if we did not do it in real time.

%% --------Embedded system--------%%
Embedded system provides connected devices processing power and the ability of
executing specific tasks or applications, so the real-time processing is
available. 
%-------------- problem description-------------%

Power consumption is among the biggest challenges targeting connected  objects,
particularly in the industrial domains, where several sensing  systems are
commonly launched in the field to run for days or even  weeks without being
recharged. Typically, such devices use sensors to  capture properties such as
temperature or motion, and stream them to a  host system over a radio
transmission protocol such as Bluetooth  Low-Energy (BLE). System designers aim
to reduce the rate of data  transmission as much as possible, as radio
transmission is a  power-hungry operation.

% ------------------ goal motivation -----------------%

% 我们的目标是利用data stream summarization 减少传输数据
% 例如 当 cardinality = 1 时,表示数据 时不变的, 从而我们只需要传输1个数据而不是所有数据 从而减少数据的传输

The goal of this thesis is to reduce the rate of data transmission through data
stream summarization. For instance, Only one data point will be transmitted
rather than all data points if the cardinality of the stream equals 1, which
means the streaming data are constant.


% ----------------------- Neblina --------------------%
Motsai's Neblina module, a system with a Nordic Semiconductor nRF52832
micro-controller, 64~KB of RAM,  and Bluetooth Low Energy connectivity. Neblina
has a 3D  accelerometer, a 3D gyroscope, a 3D magnetometer, and environmental
sensors for humidity, temperature and pressure. The platform is  equipped with
sensor fusion algorithms for 3D orientation tracking and  a machine learning
engine for complex motion analysis and motion  pattern
recognition~\cite{sarbishei2016accuracy}.




% ------- outline and contribution(publication)-------%



% LTC summary
The Lightweight Temporal Compression method 
(LTC~\cite{schoellhammer2004lightweight}) has been designed 
specifically for energy-constrained systems, initially sensor networks. 
It approximates data points by a piece-wise linear function that 
guarantees an upper bound on the reconstruction error, and a reduced 
memory footprint in $\mathcal{O}(1)$. However, LTC has only been 
described for 1D streams, while streams acquired by connected objects, such as 
acceleration or gyroscopic data, are often multi-dimensional. 


%% contribution 

The contribution of this thesis is given as follow:
\begin{enumerate}
  \item Formalize the description of original LTC algorithm
  \item Extend LTC to dimension $n$.
  \item Propose an algebraic formulation of n-dimensional LTC algorithm.
  \item Introduce an norm-independent expression of n-dimensional LTC, according to
  the algebraic formulation.
\end{enumerate}

% In this paper, we extend LTC to dimension $n$. To do so, we propose an 
% algebraic formulation of the algorithm that also yields a 
% norm-independent expression of it.

%% internship and publish
I did 4 months internship in Motsai from Sep. 2019 to Dec. 2019, and I implement
our extension on  Motsai's Neblina module
\footnote{\url{https://motsai.com/products/neblina}}, and we test it on 3D
acceleration streams acquired during human exercises, namely biceps curling,
walking and  running. Our paper "A multi-dimensional extension of the
Lightweight Temporal Compression method " was published~\cite{li2018multi} in
IEEE Big Data conference 2018. Our implementation of LTC is available as free
software.



% define transmitted, received points, compression ratio
In this thesis I assume that the stream consists of a sequence of data points
received at uneven intervals. The compression algorithm  \emph{transmits} fewer
points than it receives. The transmitted points  might be included in the
stream, or computed from stream points. The  \emph{compression ratio} is
calculated by (1-$\frac{1}{C_R}$) where $C_R$ is the ratio between the number of
received points and the number of transmitted points. An application
reconstructs the stream from the transmitted points: the  \emph{reconstruction
error} is the maximum absolute difference between  a point of the reconstructed
stream, and the corresponding  point in the original stream.


% Outline
Chapter~\ref{sec:relatedwork} provides some background on stream summarization,
lossless compression and lossy compression methods.
Chapter~\ref{chap:ltc-extension} formalizes the description initially proposed
in~\cite{schoellhammer2004lightweight} and presents our norm-independent
extension to dimension $n$, together with description of our implementation in
Section~\ref{sec:implementation}. Chapter~\ref{chap:expsAndResults} reports on
experiments to validate our  implementation, and evaluates the impact of
n-dimensional LTC on  energy consumption of connected objects.





% \todo{add follow content into Chapter 2}
% Compression is a key technique to reduce the rate of radio 
% transmission.  While in several applications lossless compression 
% methods are more desirable than lossy compression techniques, in the 
% context of IoT and sensor data streams, the measured sensor data 
% intrinsically involves noise and measurement errors, which can 
% be treated as a configurable tolerance for a lossy compression algorithm. 

% Resource-intensive lossy compression algorithms such as the ones based on 
% polynomial interpolation, discrete cosine and Fourier transforms, or 
% auto-regression methods~\cite{lu2010optimized} are not well-suited for 
% connected objects, due to the limited memory available on 
% these systems (typically a few KB), and the energy consumption 
% associated with CPU usage. Instead, compression algorithms need 
% to find a trade-off between reducing network communications and 
% increasing memory and CPU usage. As 
% discussed in~\cite{zordan2014performance}, linear compression methods 
% provide a very good compromise between these two factors, leading to 
% substantial energy reduction.



% data summarization 可以提取数据特征
\todo{write summarization, or move this to Section 2.1}
% \url{https://www-users.cs.umn.edu/~kumar001/papers/varun_icdm05-summ.pdf}

% So some previous research has discussed various stream analytic
% problems, such as estimating cardinality, frequency moments and clustering.
