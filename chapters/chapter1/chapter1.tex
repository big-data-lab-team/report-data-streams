\chapter{Introduction}

% \TG{At the end of the introduction, you should mention that the work
% described in Ch3 and 4 was published and give the reference.}
%------------------------------------------------------------------------------%
%IoT, Embedded system, sensors,
%Motsai(internship coding), Neblina, Big Data, Stream

% context
% problem description
% goal motivation
% outline and contribution(publication)

%-----------------------------------------------------------------------------%

%% structure of Introduction
                
%                 |- IoT
% - Constant -----|- WSNs sensors
%                 |- streaming data, data analysis
%                 |- Embedded system
                
% - Problem  -----|- power consuming, radio transmission is power-hungry

                
% - goal, motive -|- motivation: try to extend the work time of the Battery-powered devices 
%                 |- goal: Reduce rate of transmission of through data summarization

% - contribution -|- Formalize the description of original LTC algorithm
%                 |- Extend LTC to dimension $n$.
%                 |- Propose an algebraic formulation of n-dimensional LTC algorithm.
%                 |- Introduce an norm-independent expression of n-dimensional LTC, according 
%                 |  to the algebraic formulation.
%                 |- Internship and publish
% - outline

\section{Context}

% -------------------- context------------------------%

%% -------IoT--------%%%% --------Embedded system--------%%
With the recent technological advances of the \acrfull{iot} applications, the
number of connected devices will reach 75 billion by
2025\footnote{\url{https://www.statista.com/statistics/471264/iot-number-of-connected-devices-worldwide}}.
So far, the \acrshort{iot} has been involved in many fields such as medical
care, military, sports and industrial
manufacturing~\cite{boudargham2017exhaustive, lai2013survey, da2014internet}.
Embedded system in objects provides processing power and the ability to execute
specific tasks or applications. In industrial domains, connected objects are
often used for capturing properties such as temperature, and receiving signals
or data from other devices. In domestic domains, many household products are
expected to provide more functions that could make human activities more
convenient and improve the quality of life. For example, people can remotely
control household equipment with smart electronics and embedded systems. To
deploy IoT-based products and services, many \acrshort{iot} technologies have
been utilized~\cite{lee2015internet}:
\begin{itemize}
    \item \acrfull{rfid}: identify objects automatically
    and capture its data using radio waves, a tag and a reader.
    \item Middleware: makes communication and input/output easier for software
    developers
    \item Cloud computing: provides a pathway to handle and process the massive
    amounts of data generated by \acrshort{iot}.
    \item Wireless sensor networks: measure, monitor and record the physical
    or environmental conditions.
\end{itemize}

%% -----WSNs----%% 
The most typical networks of connected objects are \acrfull{wsn}. \acrshort{wsn}
consist of autonomous sensor-equipped devices to monitor and sense the physical
or environmental variables of our world~\cite{lee2015internet, li2016temporal}.
\acrshort{wsn} can be deployed with \acrshort{rfid} systems to obtain more
accurate measures, for instance, temperature, movements, and
location~\cite{lee2015internet, atzori2010b}. Sensor devices are connected by
low energy wireless networks such as \emph{\acrlong{ble}}
(\acrshort{ble})\footnote{\url{https://web.archive.org/web/20170310111443/https://www.bluetooth.com/what-is-bluetooth-technology/how-it-works/low-energy}}
and \emph{IEEE 802.15.4}\footnote{\url{http://www.ieee802.org/15/pub/TG4.html}}
and send the streaming data to sink sensor nodes or heavier clients.

%% -----data stream, big data--------%%
Streaming data is produced by connected objects in \acrshort{iot} and
transferred over the network. Different from an offline data-set, a data stream
is a data model in which large volumes of data arrive continuously and cannot be
saved completely~\cite{o2002streaming}. The data point in the data stream can
only be received in the order, and it is impossible to random access the
data~\cite{o2002streaming}.

With the rise of data science, the data become more and more important as it can
provide knowledge and information after filtering and learning. The data stream
is an important way to collect data, enabling data science. The large volume and
rapid velocity of data stream make us have to handle or save streaming data
timely. We may lose the opportunity to process the data at all, if we do not do
it in real time. This creates algorithmic challenges, related to the restriction
of memory and computing power in connected objects~\cite{o2002streaming}. Some
previous research has reviewed problems in streaming data analysis, such as
finding frequent elements, estimating quantiles or path
analysis~\cite{kejariwal2015real}.

In general, there are two different types of sensor streams, (1) one-dimensional
stream, for instance, temperature, humidity, and pressure. (2) multi-dimensional
stream, such as accelerometer, gyroscope, and magnetometer. The former is quite
common in our daily life, and much research has been conducted with
them~\cite{kulwicki1991humidity, oprea2009temperature, woyessa2016temperature}.
In contrast, the latter often exits in the industrial field and has not been
widely studied in literature.



% \english{Thus, the
% streaming data produced in \acrshort{iot} also important, but because of \TG{``'cause is a contraction of because and
% cannot be used in written language''} the characteristic of
% streaming data, it is needed to process streaming data timely due to its volume
% % and velocity \TG{sentence is too long, confusing}}. 
% Embedded system provides connected devices processing power and
% the ability of executing specific tasks or applications, so the real-time
% processing is available.

%-------------- problem description-------------%

In the field of \acrshort{iot} networks, power consumption is among the biggest
challenges targeting connected objects, particularly in industrial domains,
where several sensing systems are commonly launched in the field to run for days
or even weeks without being recharged. Typically, such devices use sensors to
capture properties such as temperature or motion, and stream them to a host
system over a radio transmission protocol such as \acrfull{ble}. With the
increasing computing requirements (such as data analysis) in embedded systems,
connected objects that have a small capacity of the battery cannot operate for a
long time. To extend the lifetime of objects without decreasing computing power,
system designers aim to reduce the rate of data transmission as much as
possible, as radio transmission is a power-hungry operation.


\section{Goal of the thesis}
% ------------------ goal motivation -----------------%

Reducing the rate of data transmission is a good way to decrease the power
consumption in connected objects. In this thesis, we intend to find or discover
a data summarization method which is able to help us to shorten the length of
the stream transmitted and retain the important information. Furthermore, the
method has to deal with the limitations of memory and processing power in
connected objects and the multi-dimensional streams from sensors. In other
words, this method must be non-resource-intensive, and it can work for both
one-dimensional and multi-dimensional streams.
% The limitations of memory and processing power in connected objects require that the summarization methods must   
% To decrease energy consumption, a good way is reducing the rate of In thesis, 
% The goal of this thesis is to reduce the rate of data transmission through data
% stream summarization \TG{This one-sentence goal should also mention connected objects and nD vs 1D. This is the most
% important sentence of the introduction}.
Summarization can be considered as a transformation of the original data to
smaller summaries or patterns which contain as much information as possible.
With data stream summarization, we might reduce the rate of transmitted data.
For instance, only one data point will be transmitted rather than all data
points, if all the elements in the stream are identical. 

Data compression is one data summarization technique, representing the data
stream into a compacted version. Many data compression algorithms have been
proposed for text compression~\cite{shanmugasundaram2011comparative,
sayood2017introduction} and image compression~\cite{shum2003survey,
zaineldin2015image}, but most previous algorithms are unable to fit into
connected objects because of their limited computation power and memory. In
addition, the compression process also costs energy. Thus, we also need to
compare the energy cost of compression and the saved energy from reducing the
rate of data transmitted. As more computation requires more
energy~\cite{pope2018accelerometer}, a compression algorithm with low
computational complexity is needed.  Meanwhile, the common way to handle
multi-dimensional ($n$-dimensional) data stream is compressing each parameter
respectively, just the same as compress $n$ one-dimensional data streams at the
same time. But the parameters in multi-dimensional data point are dependent on
each other, we have to consider them as a whole and process them together.
Overall, we aim at finding a compression algorithm that adapts to
multi-dimensional data streams and requires low computational and time
complexity to reduce the size of transmitted data.

% \TG{You should also mention nD compression vs 1D}

% ----------------------- Neblina --------------------%
In our experiments, we test compression methods in Motsai's Neblina module, a
system with a Nordic Semiconductor nRF52832 micro-controller, 64~KB of RAM, and
\acrlong{ble} connectivity. Neblina has a 3D accelerometer, a 3D
gyroscope, a 3D magnetometer, and environmental sensors for humidity,
temperature, and pressure. The platform is equipped with sensor fusion algorithms
for 3D orientation tracking and a machine learning engine for complex motion
analysis and motion pattern recognition~\cite{sarbishei2016accuracy}.


\section{Outline and contributions}

% ------- outline and contribution(publication)-------%

% LTC summary
% The Lightweight Temporal Compression method
% (LTC~\cite{schoellhammer2004lightweight}) has been designed  specifically for
% energy-constrained systems, initially sensor networks.  It approximates data
% points by a piece-wise linear function that  guarantees an upper bound on the
% reconstruction error, and a reduced  memory footprint in $\mathcal{O}(1)$.
% However, LTC has only been  described for 1D streams, while streams acquired by
% connected objects, such as  acceleration or gyroscopic data, are often
% multi-dimensional. \TG{This is out of place}


%% contribution 

The contribution of this thesis is given as follow:
\begin{enumerate}
  \item Formalize the description of original \acrshort{ltc} algorithm.
  \item Extend \acrshort{ltc} to dimension $n$.
  \item Propose an algebraic formulation of n-dimensional \acrshort{ltc}
  algorithm.
  \item Introduce an norm-independent expression of n-dimensional
  \acrshort{ltc}, according to the algebraic formulation.
  \item Validate the behavior of LTC n-dimension.
  \item Measure the impact of LTC n-dimension on energy consumption.
  \item Compare LTC n-dimension with Polynomial regression compression method.
%   \TG{You should also add contributions for the experimental work: comparison with other solutions, energy measures on Neblina.}
\end{enumerate}

Our implementation of LTC n-dimension is available as free software in
\url{https://github.com/big-data-lab-team/stream-summarization} under MIT
license, and it has already been implemented into Motsai's Neblina module
\footnote{\url{https://motsai.com/products/neblina}} during my internship at
Motsai from September 2018 to December 2018.  Moreover, the extension of
\acrshort{ltc} is included in a data stream algorithm library named
``OrpailleCC'' which can access at
\url{https://github.com/big-data-lab-team/OrpailleCC}.
% In this paper, we extend LTC to dimension $n$. To do so, we propose an 
% algebraic formulation of the algorithm that also yields a 
% norm-independent expression of it.

% % define transmitted, received points, compression ratio
% In this thesis I assume that the stream consists of a sequence of data points
% received at uneven intervals. The compression algorithm \emph{transmits} fewer
% points than it receives. The transmitted points might be included in the
% stream, or computed from stream points. The \emph{compression ratio} is
% calculated by (1-$\frac{1}{C_R}$) where $C_R$ is the ratio between the number of
% received points and the number of transmitted points. An application
% reconstructs the stream from the transmitted points: the \emph{reconstruction
% error} is the maximum absolute difference between a point of the reconstructed
% stream, and the corresponding point in the original stream. \TG{I think this should remain in Ch3.}

% % define energy save

% Outline
In the rest of this thesis, Chapter~\ref{sec:relatedwork} provides some
background on stream summarization, lossless compression, and lossy compression
methods. Chapter~\ref{chap:ltc-extension} formalizes the description of the
\acrshort{ltc} algorithm initially proposed
in~\cite{schoellhammer2004lightweight} and presents our norm-independent
extension to dimension $n$ and its implementation.
Chapter~\ref{chap:expsAndResults} reports on experiments to validate our
implementation, and evaluates the impact of n-dimensional \acrshort{ltc} on
energy consumption of connected objects.

%% internship and publish
The contents of Chapter~\ref{chap:ltc-extension} and
Section~\ref{sec:ltc-n-dimensions} in Chapter~\ref{chap:ltc-extension} are
included in our paper ``A multi-dimensional extension of the Lightweight
Temporal Compression method''~\cite{li2018multi}, published in IEEE Big Data
conference 2018.

% data summarization 可以提取数据特征
% \url{https://www-users.cs.umn.edu/~kumar001/papers/varun_icdm05-summ.pdf}

% So some previous research has discussed various stream analytic
% problems, such as estimating cardinality, frequency moments and clustering.
