\chapter{Introduction}

% \TG{At the end of the introduction, you should mention that the work
% described in Ch3 and 4 was published and give the reference.}
%------------------------------------------------------------------------------%
%IoT, Embedded system, sensors,
%Motsai(internship coding), Neblina, Big Data, Stream

% context
% problem description
% goal motivation
% outline and contribution(publication)

%-----------------------------------------------------------------------------%

%% structure of Introduction
                
%                 |- IoT
% - Constant -----|- WSNs sensors
%                 |- streaming data, data analysis
%                 |- Embedded system
                
% - Problem  -----|- power consuming, radio transmission is power-hungry

                
% - goal, motive -|- motivation: try to extend the work time of the Battery-powered devices 
%                 |- goal: Reduce rate of transmission of through data summarization

% - contribution -|- Formalize the description of original LTC algorithm
%                 |- Extend LTC to dimension $n$.
%                 |- Propose an algebraic formulation of n-dimensional LTC algorithm.
%                 |- Introduce an norm-independent expression of n-dimensional LTC, according 
%                 |  to the algebraic formulation.
%                 |- Internship and publish
% - outline

\section{Context}

% -------------------- context------------------------%

%% -------IoT--------%%%% --------Embedded system--------%%
With the recent technological advances of the \acrfull{iot} applications, the
number of connected devices will reach 75 billion by
2025\footnote{\url{https://www.statista.com/statistics/471264/iot-number-of-connected-devices-worldwide}}.
So far, the \acrshort{iot} has been involved in many fields such as medical
care, military, sports and industrial
manufacturing~\cite{boudargham2017exhaustive, lai2013survey, da2014internet}.
Systems embedded in connected objects provide processing power and the ability
to execute specific tasks or applications. In industrial domains, connected
objects are often used for capturing properties such as temperature, and
receiving signals or data from other devices. In domestic domains, many
household products are expected to provide functions that could make human
activities more convenient and improve the quality of life. For example, people
can remotely control household equipment with smart electronics and embedded
systems. To deploy IoT-based products and services, many \acrshort{iot}
technologies have been utilized~\cite{lee2015internet}:
\begin{itemize}
    \item \acrfull{rfid}: identifies objects automatically
    and captures data using radio transmission, a tag and a reader.
    \item \acrfull{wsn}: measure, monitor and record the physical
    or environmental conditions. 
    \item Middleware: makes communication and input/output between software applications easier for software
    developers.
    \item Cloud computing: provides a infrastructure to handle and process the massive
    amounts of data generated by \acrshort{iot}.
    % \TG{Put it before cloud or middleware, so that the list goes 
    % from bottom to top.}
\end{itemize}

%% -----WSNs----%% 
The most typical networks of connected objects are \acrlong{wsn}. It
consists of autonomous sensor-equipped devices to monitor and sense the physical
or environmental variables of our world~\cite{lee2015internet, li2016temporal}.
\acrshort{wsn} can be deployed with \acrshort{rfid} systems to obtain more
accurate measures, for instance, temperature, movements, and
location~\cite{lee2015internet, atzori2010b}. Sensor devices are connected by
low energy wireless networks such as \emph{\acrlong{ble}}
(\acrshort{ble})\footnote{\url{https://web.archive.org/web/20170310111443/https://www.bluetooth.com/what-is-bluetooth-technology/how-it-works/low-energy}}
and \emph{IEEE 802.15.4}\footnote{\url{http://www.ieee802.org/15/pub/TG4.html}},
and they send the streaming data to sink sensor nodes or heavier clients.

%% -----data stream, big data--------%%
Streaming data is produced by connected objects in the \acrshort{iot} and
transferred over the network. Different from an offline data-set, a data stream
is a data model in which large volumes of data arrive continuously and cannot be
saved completely~\cite{o2002streaming}. Data points in a data stream can
only be received in order, and it is impossible to randomly access the
data~\cite{o2002streaming}.

With the rise of data science, data becomes increasingly important as it can
provide knowledge and information after filtering and learning. Data streams are
an important way to collect data and thus enable data science. The large volume
and rapid velocity of data streams require that systems handle or save streaming
data in a timely manner. We may lose the opportunity to process the data at all,
if we do not do it in real time. This requirement creates algorithmic
challenges, related to the restriction of memory and computing power in
connected objects~\cite{o2002streaming}. Some previous research has reviewed
problems in streaming data analysis, such as finding frequent elements,
estimating quantiles, or detecting patterns in data
stream~\cite{kejariwal2015real}.
%\TG{path analysis is not clear, I would choose another example.}.

In general, there are two different types of sensor streams: (1) one-dimensional
stream, for instance, temperature, humidity, and pressure, and (2)
multi-dimensional streams, for instance 3D streams such as accelerometer,
gyroscope, and magnetometer. One-dimensional streams are quite common in our
daily life, and much research has been targeting
them~\cite{kulwicki1991humidity, oprea2009temperature, woyessa2016temperature}.
In contrast, multi-dimensional streams have not been widely studied in the
literature, although they are equally popular.

% \english{Thus, the
% streaming data produced in \acrshort{iot} also important, but because of \TG{``'cause is a contraction of because and
% cannot be used in written language''} the characteristic of
% streaming data, it is needed to process streaming data timely due to its volume
% % and velocity \TG{sentence is too long, confusing}}. 
% Embedded system provides connected devices processing power and
% the ability of executing specific tasks or applications, so the real-time
% processing is available.

%-------------- problem description-------------%

In the field of \acrshort{iot} networks, power consumption is among the biggest
challenges targeting connected objects, particularly in industrial domains,
where several sensing systems are commonly launched in the field to run for days
or even weeks without being recharged. Typically, such devices use sensors to
capture properties such as temperature or motion, and stream them to a host
system over a radio transmission protocol such as \acrfull{ble}. With the
increasing computing requirements of embedded systems, connected objects that
have a small battery capacity cannot operate for a long time while transmitting
data. To extend the lifetime of objects without decreasing computing power,
system designers aim to reduce the rate of data transmission as much as
possible, as radio transmission is a power-hungry operation.


\section{Goal of the thesis}
% ------------------ goal motivation -----------------%

Reducing the rate of data transmission is a good way to decrease the power
consumption in connected objects. In this thesis, we intend to find a data
summarization method able to shorten the length of the transmitted stream while
retaining the important information in it. Furthermore, the method has to deal
with the limitations of memory and processing power in connected objects and the
multi-dimensional streams from sensors. In other words, this method must be
non-resource-intensive, and it has to work for both one-dimensional and
multi-dimensional streams.
% The limitations of memory and processing power in connected objects require that the summarization methods must   
% To decrease energy consumption, a good way is reducing the rate of In thesis, 
% The goal of this thesis is to reduce the rate of data transmission through data
% stream summarization \TG{This one-sentence goal should also mention connected objects and nD vs 1D. This is the most
% important sentence of the introduction}.
Summarization can be considered as a transformation of the original data to
smaller summaries or patterns which contain as much information as possible.
With data stream summarization, we might reduce the rate of transmitted data.
For instance, only one data point will be transmitted rather than all data
points, if all the elements in the stream are identical. 

Data compression is a data summarization technique representing the data stream
into a compacted version. Many data compression algorithms have been proposed
for text compression~\cite{shanmugasundaram2011comparative,
sayood2017introduction} and image compression~\cite{shum2003survey,
zaineldin2015image}, but most previous algorithms are unable to fit into
connected objects because of their limited computation power and memory. In
addition, the compression process also costs energy. Thus, we also need to
compare the energy cost of compression and the saved energy from reducing the
rate of data transmitted. As more computation requires more
energy~\cite{pope2018accelerometer}, a compression algorithm with low
computational complexity is needed. Meanwhile, the common way to handle
multi-dimensional ($n$-dimensional) data streams is to compress each dimension
independently, which boils down to compressing $n$ one-dimensional data streams
at the same time. But the parameters in multi-dimensional data points are
dependent on each other, we have to consider them as a whole and process them
together. Overall, we aim at finding a compression algorithm that adapts to
multi-dimensional data streams and requires low computational and time
complexity to reduce the size of transmitted data.

% LTC summary
\acrfull{ltc}~\cite{schoellhammer2004lightweight} is one of stream compression
method which has been designed specifically for energy-constrained systems,
initially sensor networks. It approximates data points by a piece-wise linear
function that guarantees an upper bound on the reconstruction error, and a
reduced memory footprint in $\mathcal{O}(1)$. However, LTC has only been
described for 1D streams, therefore, in this paper, we extend LTC to dimension
$n$.

% ----------------------- Neblina --------------------%
In our experiments, we test compression methods in Motsai's Neblina module, a
system with a Nordic Semiconductor nRF52832 micro-controller, 64~KB of RAM, and
\acrlong{ble} connectivity. Neblina has a 3D accelerometer, a 3D gyroscope, a 3D
magnetometer, and environmental sensors for humidity, temperature, and pressure.
The platform is equipped with sensor fusion algorithms for 3D orientation
tracking and a machine learning engine for complex motion analysis and motion
pattern recognition~\cite{sarbishei2016accuracy}.


\section{Outline and contributions}

% ------- outline and contribution(publication)-------%

%% contribution 

The contributions of this thesis are the following:
\begin{enumerate}
  \item Formalize the description of original \acrshort{ltc} algorithm. 
  %\TG{LTC needs to be defined before, preferably in the context section where you talk about compression.}
  %\item Extend \acrshort{ltc} to dimension $n$.
  \item Propose an algebraic formulation of n-dimensional \acrshort{ltc}
  algorithm, and also introduce an norm-independent expression according to the formulation.
%   \item Introduce an norm-independent expression of n-dimensional \acrshort{ltc}, according to the
%   algebraic formulation. \TG{The difference between 2, 3 and 4 is unclear}
  \item Implement LTC n-dimension for Infinity and Euclidean norms.
  \item Validate the behavior of LTC n-dimension.
  \item Measure the impact of LTC n-dimension on energy consumption.
  \item Compare LTC n-dimension with Polynomial regression compression method.
%   \TG{You should also add contributions for the experimental work: comparison with other solutions, energy measures on Neblina.}
\end{enumerate}

Our implementation of LTC n-dimension is available as free software in
\url{https://github.com/big-data-lab-team/stream-summarization} under MIT
license, and it has already been implemented into Motsai's Neblina
module\footnote{\url{https://motsai.com/products/neblina}} during my internship
at Motsai from September 2018 to December 2018. Moreover, the extension of
\acrshort{ltc} is included in a data stream algorithm library named
``OrpailleCC'' available at
\url{https://github.com/big-data-lab-team/OrpailleCC} and currently under review
in the Journal of Open-Source Software.
% In this paper, we extend LTC to dimension $n$. To do so, we propose an 
% algebraic formulation of the algorithm that also yields a 
% norm-independent expression of it.

% % define transmitted, received points, compression ratio
% In this thesis I assume that the stream consists of a sequence of data points
% received at uneven intervals. The compression algorithm \emph{transmits} fewer
% points than it receives. The transmitted points might be included in the
% stream, or computed from stream points. The \emph{compression ratio} is
% calculated by (1-$\frac{1}{C_R}$) where $C_R$ is the ratio between the number of
% received points and the number of transmitted points. An application
% reconstructs the stream from the transmitted points: the \emph{reconstruction
% error} is the maximum absolute difference between a point of the reconstructed
% stream, and the corresponding point in the original stream. \TG{I think this should remain in Ch3.}

% % define energy save

% Outline
In the rest of this thesis, Chapter~\ref{sec:relatedwork} provides some
background on stream summarization, lossless compression, and lossy compression
methods. Chapter~\ref{chap:ltc-extension} formalizes the description of the
\acrshort{ltc} algorithm initially proposed
in~\cite{schoellhammer2004lightweight} and presents our norm-independent
extension to dimension $n$ and its implementation.
Chapter~\ref{chap:expsAndResults} reports on experiments to validate our
implementation, evaluates the impact of n-dimensional \acrshort{ltc} on
energy consumption of connected objects, and compares n-dimensional
\acrshort{ltc} with polynomial regression compression method.

%% publish
The contents of Chapter~\ref{chap:ltc-extension} and
Section~\ref{sec:ltc-n-dimensions} in Chapter~\ref{chap:expsAndResults} are
included in our paper ``A multi-dimensional extension of the Lightweight
Temporal Compression method''~\cite{li2018multi}, published in the 3rd
Workshop on Real-time and Stream Analytics in Big Data \& Stream Data
Management, co-located with the IEEE Big Data conference 2018.
